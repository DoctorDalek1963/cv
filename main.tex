\documentclass[a4paper, 12pt]{article}

\usepackage[UKenglish]{babel} % UK writing style
\usepackage[a4paper, total={155mm, 245mm}]{geometry} % Good margins

\usepackage{array}

\usepackage[hidelinks]{hyperref}
\hypersetup{
	pdftitle = {CV for Dyson Dyson},
	pdfauthor = {Dyson Dyson}
}

% PLEASE NOTE: The vast majority of this preamble is taken from https://latex-tutorial.com/cv-latex-guide/

\pagestyle{empty}

\usepackage{sectsty}

\sectionfont{%
	\large\bfseries
	\sectionrule{0pt}{0pt}{-5pt}{1pt}}

% Size of the boxes used to align text
\newlength{\spacebox}
\settowidth{\spacebox}{123456789}

% Vertical space separator between entries
\newcommand{\sepspace}{\vspace*{1ex}}

\newcommand{\setspacing}{\setlength{\parindent}{0em}\setlength{\parskip}{2.5ex}}

\newcommand{\name}[1]{
	\Huge
	% Name centered and bold
	\begin{center}\textbf{#1}\end{center}\par
	\normalsize}

\newcommand{\pronouns}[1]{
	\normalsize
	% Pronouns centered and slanted
	\begin{center}\textsl{#1}\end{center}\par}

\newcommand{\info}[2]{
	% Set indentation
	\noindent\hangindent=2em\hangafter=0
	% Create a box to align two pieces of text
	\parbox{\spacebox}{%
		\textsl{#1}}
	#2\par}

\newcommand{\skill}[2]{
	% set specific indentation for personal information
	\noindent\hangindent=2em\hangafter=0
	% create a box to align two pieces of text
	\parbox{3\spacebox}{
		\textbf{#1}}
	#2 \par}

\newcommand{\work}[4]{
	% Name of the job
	\noindent\textbf{#1}
	\hfill
	\framebox{%
		\parbox{14em}{%
			\centering\textbf{#2}}}\par
	% New paragraph with the place in italics
	\noindent\textsl{#3}\par
	% Description in smaller text
	\vspace*{0.5em}
	\noindent\hangindent=2em\hangafter=0\small #4
	\normalsize\par}

\begin{document}

\name{Dyson Dyson}
\vspace*{-1.5ex}
\pronouns{they/them}

\sepspace

\info{Email}{\href{mailto:dyson.dyson@icloud.com}{\texttt{dyson.dyson@icloud.com}}}
\info{Phone}{07802 844691}
%\info{GitHub}{\url{https://github.com/DoctorDalek1963}}

\section*{Profile}
\vspace{-1.5ex}
{
\setspacing
I am currently in a gap year, intending to study Mathematics at university starting in September 2024, and I am looking for a part time position until then. I am very aspirational, and I'm quick and eager to learn new skills and take on new challenges in a professional environment.

My A Level grades were good enough to get me in basically anywhere, but I wanted to go to Cambridge and I unfortunately missed the requirements on their entrance exam, so I decided to take a gap year and apply again.
}

\section*{Employment}

\work{Gardener}{July 2022 \-- August 2022}{Richmond Villages Northampton}{When working in this care village, I gained valuable experience in communication when I had to interact with the residents. They are all elderly and many of them have conditions like dementia or Alzheimer's, meaning patience and clear communication is key. I also learned about good teamwork skills, which I had to use daily with my colleagues.}
\vspace{4ex}
\work{After-school club manager (volunteer)}{January 2022 \-- June 2022}{The Duston School}{While in Year 12, I helped to run and manage an after-school maths club for younger year groups. It was a club designed to challenge students with mathematical puzzles that they had not previously seen. I had to procure the questions in advance, create PowerPoints, and help the students when they needed it.}

\section*{Education}

\begin{center}
\framebox{\parbox{6em}{%
	\centering\textbf{A Levels}\par}}
\end{center}

\begin{center}
	\begin{tabular}{m{12em} m{12em}}
		A* in Mathematics & A* in Further Mathematics \\
		A in Computer Science
	\end{tabular}
\end{center}

\begin{center}
\framebox{\parbox{6em}{%
	\centering\textbf{GCSEs}\par}}
\end{center}

\begin{center}
	\begin{tabular}{m{16em} m{10em}}
		9 in Mathematics (Higher) & 9 in Computer Science \\
		99 in Combined Science (Higher) & 8 in History \\
		7 in English Language & 7 in German (Higher) \\
		6 in English Literature
	\end{tabular}
\end{center}

%\section*{Technical skills}
%\skill{Programming languages}{Rust, Python, Java, Bash, \LaTeX}
%\skill{Other}{Markdown, Git, Neovim, Ubuntu, Linux}

\end{document}
